% Introduction
\addcontentsline{toc}{chapter}{Introduction}
\chapter*{Introduction} % Write in your own chapter title



\fancyhead[RO,LE]{\thepage}
\fancyhead[LO]{\emph{Introduction}}
\fancyhead[RE]{\emph{Introduction}}

\setlength{\parskip}{0.5pt}

\bigskip

\addcontentsline{toc}{section}{Overview}
\section*{Overview}
\large Statistical analysis is a process that can be broken into different steps. 
From data collection, through data analysis, up to the yielding of consistent results, statisticians are continuously asked to come down to compromises in the attempt of tackling the underlying trends of their object of study. 
Among these steps, the greatest controversy is probably bound to model selection: a bitter truth known to every statistician is that there is no such a thing as a best model. With that said, it is still reasonable to search - if not for the best - for a better model and, in this respect, several indexes were built for comparing different models with each other. %and to attest, mathematically, the superiority of a model over another.  WHAT ARE THE TERMS OF THE COMPARISON
A particularly powerful index is the Akaike's information criterion; it is based on the
likelihood and asymptotic properties of the maximum likelihood estimator and allows model comparison in terms of predictability and parsimony. %for a given model, it maximizes the double difference between the likelihood function of the model and the number of parameters. 
Despite being a powerful tool, its strict dependence on the likelihood implies the model distribution to be fully known: a requirement that cannot always be fulfilled. In this context, this work sets its aim at assessing methods to widen the AIC usage to those models for which there is no likelihood defined. We will specifically focus our attention on the Akaike's information criterion for models estimated through the generalized estimating equation (GEE) approach, very useful for working with correlated data, but 
based on the quasi-likelihood method, and hence, unconstrained by the assumption of a distribution. 
\noindent


\phantomsection
\addcontentsline{toc}{section}{Summary}
%\fancyhead[RE]{\emph{Main contributions of the thesis}}
\newpage
\section*{Summary}
%\noindent

