
%% ----------------------------------------------------------------
%% Thesis.tex -- MAIN FILE (the one that you compile with LaTeX)
%% ---------------------------------------------------------------- 

% Set up the document
\documentclass[a4paper, 12pt, twosided]{Thesis}  % Use the "Thesis" style, based on the ECS Thesis style by Steve Gunn
\graphicspath{{Figures/}}  % Location of the graphics files (set up for graphics to be in PDF format)
\usepackage{multicol}
% Include any extra LaTeX packages required
%\usepackage[square, numbers, comma, sort&compress]{natbib}  % Use the "Natbib" style for the references in the Bibliography
\usepackage{verbatim}  % Needed for the "comment" environment to make LaTeX comments
\usepackage{vector}  % Allows "\bvec{}" and "\buvec{}" for "blackboard" style bold vectors in maths
\usepackage{hyperref}
\usepackage[dvipsnames]{xcolor}
 \usepackage{breakurl}
\hypersetup{
%urlcolor   = {blue},
pdfauthor  = {name},
pdfsubject = {PhD thesis},
pdftitle   = {PhD thesis},
pdfcreator = {name},
colorlinks   =false,
    % citecolor    =   Cerulean,
     linkbordercolor=Salmon,
       citebordercolor=MidnightBlue
       %urlbordercolor=green
}  % Colours hyperlinks in blue, but this can be distracting if there are many links.
% Colours hyperlinks in blue, but this can be distracting if there are many links.
\usepackage[round]{natbib}
\usepackage[italian, english]{babel}
\usepackage{amsfonts}
\usepackage{amsmath}
\usepackage{amsthm}
\usepackage{bigints}
\usepackage{fancyhdr}
\usepackage[autostyle]{csquotes}  
\usepackage{psfrag}
\usepackage{fancyhdr}
\usepackage{amssymb}
\usepackage{graphicx}
\usepackage{tikz}
\usetikzlibrary{fit,positioning,calc}

\usepackage{color}
\usepackage{colortbl}
\usepackage{dashrule}
\usepackage{booktabs}
\usepackage{url}
\usepackage{color, caption, multirow}
\usepackage[final]{pdfpages}
\usepackage{indentfirst}
\usepackage{multirow}
\usepackage{color}
\usepackage{lipsum}
\usepackage{algorithm}
\usepackage{algorithmic}
\usepackage{float}
\usepackage[titletoc,title]{appendix}


\usepackage{pdfpages}

%%% TITLE
\renewcommand{\maketitle}{
\begin{titlepage}%

 \thispagestyle{empty}
    \enlargethispage{18cm}  % just to make sure LaTeX doesn't
      % generate a new page
 
    \noindent
\begin{figure}
      \centering
			\includegraphics{Figures/logo}
\end{figure}
\begin{center}
\huge Universit\`a degli Studi di Padova \\
\LARGE Dipartimento di Scienze Statistiche\\
\vspace{1cm}

\large Corso di Laurea Triennale in\\
\large Statistica per le Tecnologie e le Scienze
\end{center}

\vspace{1cm}
\begin{center}
\large Relazione finale \\
\large
{\textbf{Akaike's Information Criterion in Generalized Estimating Equations}}
\end{center}

\begin{flushleft}
\vspace{1.5cm}
\textbf{Relatore:} Prof. Alessandra Salvan\\
\vspace{0.2cm}
Dipartimento di Scienze Statistiche\\
%\vspace{0.5cm}
%\textbf{Co-supervisore:} Prof. ...\\
\end{flushleft}


\begin{flushright}
	\vspace{2cm}
	\textbf{Laureando:} Francesco Ignazio Re\\
	Matricola: 1149556
\end{flushright}
\begin{flushleft}
	\vspace{1cm}
	-- -- ----
\end{flushleft}
\end{titlepage}
}
\makeindex  


\let\Sectionmark\sectionmark
\def\sectionmark#1{\def\Sectionname{#1}\Sectionmark{#1}}


%%----------------------------------------------------------------
\begin{document}
\maketitle

\frontmatter	  % Begin Roman style (i, ii, iii, iv...) page numbering

\setcounter{page}{1}
\pagenumbering{roman}
%%----------------------------------------------------------------

\setstretch{1.3}  % It is better to have smaller font and larger line spacing than the other way round
\setlength{\parindent}{3ex}
% Define the page headers using the FancyHdr package and set up for one-sided printing
\fancyhead{}  % Clears all page headers and footers
\rhead{\thepage}  % Sets the right side header to show the page number
\lhead{}  % Clears the left side page header


\pagestyle{empty}  % Page style needs to be empty for this page
\cleardoublepage


\setstretch{1.3}  % Return the line spacing back to 1.3



\pagestyle{empty}  % Page style needs to be empty for this page


%%----------------------------------------------------------------

%%%%%%%%%%%%%%%%%%%%%%% Abstarct %%%%%%%%%%%%%%%%%%%%%%%%%

\btypeout{Abstract Page}
  \null\vspace{4cm}
  \begin{center}
    \setlength{\parskip}{0pt}
    {\huge{{\textbf{Abstract}}} \par}
        \bigskip
        \end{center}
        

        --- Draft ---
Generalized linear models are a powerful tool in statistical analysis for modeling data whose distribution belongs to the exponential family. However, even though they widen the class of doable problems, overcoming the necessity of normally distributed observations, they still set a few limitations, being themselves based on the maximum likelihood method, and hence, on the specification of an a-priori settled model.These constraints may present an obstacle when working with data whose variability is not well represented by the one assumed in the model.









\pagestyle{empty}  % Page style needs to be empty for this page
\cleardoublepage
\pagestyle{plain}
%% ----------------------------------------------------------------

\newpage
\thispagestyle{empty}
\mbox{}

%%%%%%%%%%%%%%%%%%%%%%% Put Sommario here %%%%%%%%%%%%%%%%%%%%%%%%%

\pagestyle{empty}
\btypeout{Abstract Page}
  \null\vspace{3cm}
  \begin{center}
    \setlength{\parskip}{0pt}
    {\huge{{\textbf{Sommario}}} \par}
        \bigskip
        \end{center}
\selectlanguage{italian}
Contenuto del sommario.


\selectlanguage{english}

% Sommario ended, start a new page
\newpage
\thispagestyle{empty}
\mbox{}


\newpage
\thispagestyle{empty}
\mbox{}

\topskip0pt
\vspace*{4cm}
\begin{flushright}
	\textit{\Large{{Dedication}}}
\end{flushright}
\vspace*{\fill}


\newpage
\thispagestyle{empty}
\mbox{}

\newpage
\thispagestyle{empty}
\mbox{}

\thispagestyle{empty}
\btypeout{Abstract Page}

  \begin{center}
    \setlength{\parskip}{0pt}
    {\huge{\textbf{Acknowledgements}} \par}
        \bigskip
        \end{center}
{ Acknowledgements content.}


\newpage
\thispagestyle{empty}
\mbox{}


\fancyhead[LE]{\thepage}
\fancyhead[RE]{Contents}

\pagestyle{fancy}  %The page style headers have been "empty" all this time, now use the "fancy" headers as defined before to bring them back
%%----------------------------------------------------------------

\tableofcontents  % Write out the Table of Contents
\newpage
\thispagestyle{empty}
\mbox{}

%%----------------------------------------------------------------



 \cleardoublepage\phantomsection
 \fancyhead[RO,LE]{\thepage}
 \fancyhead[RE]{List of Figures}
 \fancyhead[LO]{List of Figures}
 \pagestyle{fancy}

\listoffigures  % Write out the List of Figures
\clearpage
\pagestyle{empty}
%%----------------------------------------------------------------


\clearpage\phantomsection
\fancyhead[RO,LE]{\thepage}
\fancyhead[RE]{List of Tables}
\fancyhead[LO]{List of Tables}
\pagestyle{fancy}

\listoftables  % Write out the List of Tables

\clearpage\phantomsection
\pagestyle{empty}


%%----------------------------------------------------------------
\addtocontents{toc}{\vspace{1em}}
  % Return the page headers back to the "fancy" style
\mainmatter	  % Begin normal, numeric (1,2,3...) page numbering

% Include the chapters of the thesis, as separate files
% Just uncomment the lines as you write the chapters
\cleardoublepage\phantomsection
\pagestyle{fancy}
% Introduction
\addcontentsline{toc}{chapter}{Introduction}
\chapter*{Introduction} % Write in your own chapter title



\fancyhead[RO,LE]{\thepage}
\fancyhead[LO]{\emph{Introduction}}
\fancyhead[RE]{\emph{Introduction}}

\setlength{\parskip}{0.5pt}

\bigskip

\addcontentsline{toc}{section}{Overview}
\section*{Overview}
\large Statistical analysis is a process that can be broken into different steps. 
From data collection, through data analysis, up to the yielding of consistent results, statisticians are continuously asked to come down to compromises in the attempt of tackling the underlying trends of their object of study. 
Among these steps, the greatest controversy is probably bound to model selection: a bitter truth known to every statistician is that there is no such a thing as a best model. With that said, it is still reasonable to search - if not for the best - for a better model and, in this respect, several indexes were built for comparing different models with each other. %and to attest, mathematically, the superiority of a model over another.  WHAT ARE THE TERMS OF THE COMPARISON
A particularly powerful index is the Akaike's information criterion; it is based on the
likelihood and asymptotic properties of the maximum likelihood estimator and allows model comparison in terms of predictability and parsimony. %for a given model, it maximizes the double difference between the likelihood function of the model and the number of parameters. 
Despite being a powerful tool, its strict dependence on the likelihood implies the model distribution to be fully known: a requirement that cannot always be fulfilled. In this context, this work sets its aim at assessing methods to widen the AIC usage to those models for which there is no likelihood defined. We will specifically focus our attention on the Akaike's information criterion for models estimated through the generalized estimating equation (GEE) approach, very useful for working with correlated data, but 
based on the quasi-likelihood method, and hence, unconstrained by the assumption of a distribution. 
\noindent


\phantomsection
\addcontentsline{toc}{section}{Summary}
%\fancyhead[RE]{\emph{Main contributions of the thesis}}
\newpage
\section*{Summary}
%\noindent

 % Introduction

% Chapter 1
\chapter{ Models based on Maximum Likelihood Estimation }

\fancyhead[RO,LE]{\thepage}
\fancyhead[LO]{Chapter 1 - \emph{Title of chapter}}
\fancyhead[RE]{Section \thesection \ - \emph{\Sectionname}}

\setlength{\parskip}{0.5pt}

\bigskip
\section{Introduction}
\large In this chapter, we will first introduce the likelihood function along with its main properties. We will then briefly discuss Linear Models (LM) and Generalized Linear Models (GLM), as being two classes of models that use the likelihood function for the estimation of their parameters of interest. The information herein provided is referenced from ... ... ... %"Likelihood Methods in Statistics" "Thomas A. Severini"

\section{Likelihood} 
\subsection{Model Specification}

The aim of statistical inference is to gain insight regarding the underlying distribution of a phenomenon of interest $Y$, given that we have access to a limited sample of observations of $Y$, $(y_1,y_2,...,y_n)$. Assuming that $Y$ is defined by the parametric density function $f(y,\theta_0)$, with $\theta_0$ being the only unknown component of $f(\cdot)$, then our goal is to draw conclusions regarding the value $\theta_0$, using the information embedded in the sample $(y_1,y_2,...,y_n)$. In this way, we restrict our interest on a precise family of distributions to which we refer to as our model of interest. Formally, we define a parametric model $\mathcal{F}$ as

$$ \mathcal{F} = \{ f(y;\theta): \theta \in \Theta \subseteq	{\mathbb{R}}^p \} $$

with $p \in {\mathbb{N}}^+$ and $\Theta$ being the parametric space, namely the space containing all the possible values of $\theta$ and, indeed, ${\theta}_0$ itself.





 
\noindent

\subsection{Likelihood Function}
The concept of likelihood is at the very core of traditional statistical inference. The term was firstly used by Fisher, in 1921, and defined as follows:

\bigskip

\textit{The likelihood that any parameter (or set of parameters) should have any assigned value (or set of values) is proportional to the probability that if this were so, the totality of observations should be that observed.} 

\bigskip
In other words, it establishes a method to discriminate among different values of $\theta$, considering for each $\theta \in \Theta$ the values assumed by the density function conditioned to the sample $(y_1,y_2, ... , y_n)$.
Assuming the model $\mathcal{F}$ with density function $f(y,\theta)$ to be correct for the sample $(y_1,y_2, ... , y_n)$ , we can then define the likelihood function $L : \Theta \rightarrow {\mathbb{R}}^p$ as 

$$L(\theta) = L(\theta;y) = c(y)f(y;\theta), $$

with $c(y)$ being a function of the data, independent from the parameter. With respect to the model $\mathcal{F}$, the likelihood is a class of functions equivalent to each other, and differing only for the component $c(y)$. If the observations $(y_1,...,y_n)$ are independent and identically distributed, then the likelihood function  is simply the product of the individual densities, thus can be expressed as

$$L(\theta) = \prod_{i=1}^{i=n} f_{Y_i}(y_i,\theta),$$

with $f_{Y_i}(y_i,\theta)$ being the density function of the random variable $Y_i$, generator of the $i$-th observation, $y_i$, of the sample $(y_1,...,y_n)$.

For a more straightforward approach in calculations, we usually operate with the natural logarithm of the likelihood function: being the natural logarithm a monotonically increasing transformation, it does not alter the information embedded in the data, while still providing a much more manageable form.
We then define the log-likelihood as 

$$l(\theta) = l(\theta; y) =$$ log$$L(\theta; y)$$

In the case of independent and identically distributed observations, the log-likelihood would be

$$ l(\theta) = \sum_{i=1}^{i=n} l(\theta;y_i)$$







\section{Linear Models}
\noindent

\subsection{}
\noindent


\subsection{Title of subsection}
\noindent


\subsection{Title of subsection}
\noindent

\begin{table}[b]\centering\vspace{0.5cm}
	\caption{\label{tab:MLfit} ML fit of the Gamma regression model with log-link and Wald 0.95 confidence intervals for the parameters.}
	\medskip	
	\begin{tabular}{cccc}
		\toprule
		& Estimate & Estimated Standard Error & 0.95 Confidence Interval \\
		\midrule
		$\beta_1$ & 0.361 & 0.250 & (-0.128,  0.851) \\ 
		
		$\beta_2$ & 1.507 & 0.170 & (1.174, 1.839)\\
		
		$\beta_3$ & 1.859 & 0.165 & (1.535, 2.183)\\
		
		$\phi$ & 0.223 & 0.079 & (0.069, 0.377)\\
		\bottomrule
	\end{tabular}
\end{table}

\section{Generalized Linear Models}
\noindent

 

% Chapter 2
\chapter{Quasi-Likelihood Models}

\fancyhead[RO,LE]{\thepage}
\fancyhead[LO]{Chapter 2 - \emph{Title of chapter}}
\fancyhead[RE]{Section \thesection \ - \emph{\Sectionname}}

\setlength{\parskip}{0.5pt}

\bigskip

\section{Quasi-likelihood inference} 
\noindent

\section{Quasi-likelihood function}
\noindent

\subsection{Generalized Estimating Equations}
\noindent


\subsection{Title of subsection}
\noindent


\subsection{Title of subsection}
\noindent

\begin{figure}[!h]\centering
	\includegraphics[width=12cm]{qq-clot.pdf}
	
	\caption{\label{qq-clot}Normal Q-Q plots based on 2000 values of $\widehat{T}^4$ and $\widehat{T}^{4,*}$ computed under the null hypothesis $H_0\!:\beta_4=\beta_{04}$ in the \emph{clotting} example.}
\end{figure}

\section{Title of section}
\noindent

% Chapter 3
\chapter{Title of chapter}

\fancyhead[RO,LE]{\thepage}
\fancyhead[LO]{Chapter 3 - \emph{Title of chapter}}
\fancyhead[RE]{Section \thesection \ - \emph{\Sectionname}}

\setlength{\parskip}{0.5pt}

\bigskip

\section{Title of section} 
\noindent
\cite{azzalini01}

\section{Title of section}
\noindent
\cite{bart53}

\subsection{Title of subsection}
\noindent
\cite{brglm2}

\subsection{Title of subsection}
\noindent
\cite{staf92}

\subsection{Title of subsection}
\noindent
\cite{dicicciostern93}

\section{Title of section}
\noindent

\cleardoublepage
%%----------------------------------------------------------------
% Now begin the Appendices, including them as separate files

\addtocontents{toc}{\vspace{1em}} % Add a gap in the Contents, for aesthetics
\appendixtocoff
\appendixtitleon

\pagestyle{fancy}

% Cue to tell LaTeX that the following 'chapters' are Appendices
\begin{appendices}
	\renewcommand\thechapter{}

\input{./Appendices/AppendixA}	% Appendix Title
\end{appendices}
\clearpage
\pagestyle{empty}

\addtocontents{toc}{\vspace{1em}}  % Add a gap in the Contents, for aesthetics

\backmatter
%% ----------------------------------------------------------------

\bibliographystyle{CUP}
\pagestyle{fancy}
\label{Bibliography}
\fancyhead[RO,LE]{\thepage}
\fancyhead[LO]{Bibliography}
\fancyhead[RE]{Bibliography}
\bibliography{References/references}

\newpage
\thispagestyle{empty}
\mbox{}

\pagestyle{empty}
\includepdf[pages={1}, scale=1, offset=-3cm -2cm]{CVthesis/cv.pdf}
\includepdf[pages={2}, scale=1, offset=-1.75cm -1.2cm]{CVthesis/cv.pdf}
\includepdf[pages={3}, scale=1, offset=3cm -2cm]{CVthesis/cv.pdf}
\includepdf[pages={4}, scale=1, offset=-1.75cm -1.2cm]{CVthesis/cv.pdf}
\end{document}  % The End
%%----------------------------------------------------------------